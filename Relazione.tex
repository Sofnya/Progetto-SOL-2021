\documentclass[11pt]{article}
\usepackage[margin=2cm]{geometry}
\usepackage{times}
\usepackage{hyperref}
\title{Relazione Progetto SOL 2020-2021}

\date{07/09/2022}
\author{Sofia Pisani \\ Matricola: 646301}
\begin{document}


\maketitle
\pagenumbering{gobble}
\newpage
\tableofcontents
\newpage
\pagenumbering{arabic}
\begin{flushleft}
\section{Istruzioni per l'uso}
\subsection{Compilazione}
Il progetto non include eseguibili precompilati. Prima di utilizzarlo bisogna utilizzare uno dei seguenti comandi make:
\begin{description}
\item[make all] : Genera gli eseguibili server.out e client.out nella cartella out.
\item[make server] : Genera l'eseguibile server.out nella cartella out.
\item[make client] : Genera l'eseguibile client.out nella cartella out.
\item[make clean] : Ripulisce la cartella out e la cartella obj dai moduli oggetto compilati.
\item[make test\lbrack 1,2,3\rbrack]: Genera gli eseguibili necessari ed esegue automaticamente il test richiesto.
\end{description}
\subsection{Client}
\subsubsection{Avvio}
Per lanciare il client, dopo averlo compilato, eseguire il file client.out, dandogli delle opzioni valide.
\subsubsection{Comandi}
Di seguito una lista dei comandi disponibili e i loro effetti:
\begin{description}

\item[-h] : Stampa una lista dei comandi disponibili e i loro effetti.

\item[-f filename]: Connette il client al socket di nome filename.

\item[-w dirname\lbrack ,n=0\rbrack]: Scrive sul server fino a n file dalla cartella dirname, visitando ricorsivamente le subdirectory. Se n=0 o non è specificato scrive tutti i file trovati.

\item[-W file1\lbrack ,file2\rbrack]: Scrive sul server tutti i file specificati.

\item[-D dirname]: Imposta la cartella in cui il client salvera i file espulsi dal server a seguito di un capacity miss. Se non impostata i file espulsi dal server verrano ignorati.

\item[-r file1\lbrack ,file2\rbrack]: Legge dal server tutti i file specificati.

\item[-R \lbrack N=0\rbrack]: Legge n file qualsiasi dal server. Se n=0 o non è specificato scrive tutti i file trovati.

\item[-d dirname]: Imposta la cartella in cui il client salvera i file letti. Se non impostata i file letti verranno ignorati.

\item[-t time]: Imposta il tempo in millisecondi che intercorrerà tra una richiesta al server e la prossima (0 di default).

\item[-l file1\lbrack ,file2\rbrack]: Ottiene una lock sui file specificati.

\item[-u file1\lbrack ,file2\rbrack]: Rilascia la lock sui file specificati.

\item[-c file1\lbrack ,file2\rbrack]: Rimuove tutti i file specificati dal server.

\item[-p]: Abilita le stampe sullo stdout.

\end{description}


\subsection{Server}
\subsubsection{Avvio}

Per lanciare il server, dopo averlo compilato, eseguire il file server.out, passandogli opzionalmente da linea di comando un path al file di configurazione da usare. Se non specificato cerchera un file config.txt nella directory corrente, e se non trovato usera delle impostazioni di default.

\subsubsection{File di Configurazione}

Un file di configurazione valido è una serie di coppie CHIAVE=VALORE, ognuna su una linea diversa.
Il file di configurazione può anche includere commenti, cioè linee che iniziano con // verranno ignorate. Le chiavi disponibili, insieme a una breve descrizione dei loro effetti e ai loro valori di default, possono essere trovati nel file config.txt generato durante la compilazione del server.


\section{Architettura del Server}

Il client ha un architettura estremamente semplice, limitandosi a parsare le opzioni da linea di comando e lanciando le richieste necessarie al server tramite l'API. Per questo non è particolarmente interessante, e ci limiteremo a parlare dell' architettura del server.

Il server ha infatti una struttura ben più complessa, dovendo gestire i file memorizzati al proprio interno, e connessioni simultanee da più client. 

\subsection{Layout dei Thread}
Il server funziona su multipli thread:

\subsubsection{Main Thread}
Il main si occupa dello startup del server, facendo il parsing del file di config, inizializzando appropriatamente le strutture dati necessarie, e inizializzando la ThreadPool. Una volta fatto questo andra ad accettare connessioni, e passarle alla ThreadPool perchè le gestisca.
\\Infine, gestisce la terminazione intercettando i segnali SIGINT SIGHUP e SIGQUIT alla cui ricezione provvedera a liberare le risorse e a stampare il sunto delle statistiche.

\subsubsection{ThreadPool Manager}
Il manager della ThreadPool è un thread che gestisce tale struttura. Si occupa principalmente di generare e uccidere thread a seconda delle necessità, oltre a svolgere un ruolo nella terminazione della ThreadPool. I thread generati al suo interno si occupano poi indipendentemente di consumare la queue di task submittate alla ThreadPool.

\subsubsection{Worker Thread}
I worker thread si occupano di consumare le task submittate alla ThreadPool. Il server genererà una task per ogni connessione, e quindi i worker thread andranno a occuparsi della gestione di un intera connessione alla volta, secondo il seguente loop: 
\begin{itemize}

\item Ricezione della richiesta.

\item Processing della richiesta, facendo le adeguate chiamate al FileSystem e generando una risposta adeguata.

\item Invio della risposta.

\end{itemize}

I worker thread vanno inoltre a tenere traccia dello stato della loro connessione corrente, tenendo aggiornata una struttura ConnState.

\subsection{Stato del server}
Il server deve tenere traccia di:
\begin{description}

\item[Informazioni relative ai file memorizzati.] Vengono conservate nella struttura FileSystem, e gestite tramite le funzioni che usano questa struttura.

\item[Informazioni relative alle singole connessioni.] Ogni thread gestisce una connessione singola, e ne ricorda lo stato in una struttura ConnState. Informazioni rilevanti sono ad esempio i file aperti al momento e il file lockato se presente.

\item[Informazioni relative ai thread in esecuzione.] Queste vengono gestite autonomamente dalla ThreadPool.

\end{description}

Queste tre strutture dati sono le più complesse e interessanti, e sono quindi quelle di cui parleremo nel dettaglio.

\subsubsection{FileSystem}
Il FileSystem deve tenere traccia di:
\begin{description}

\item[I File presenti nella memoria.] Vengono conservati in un HashTable per permettere accessi efficienti.

\item[La capacità corrente del server.] Il numero corrente di File nel FileSystem e la dimensione totale di questi vengono conservati in due AtomicInt.

\item[I metadati dei File.] Questi servono ad implementare delle policy informate per la gestione dei capacity miss. Vengono conservati in una List.

\end{description}

Per la gestione della concorrenza, abbiamo un R/W lock su tutto il FileSystem, e una mutex per gli accessi alla lista dei metadati. L'implementazione dell HashTable è inoltre thread-safe, permettendo accessi concorrenti.

Un File è visto come un buffer di una certa dimensione, identificato univocamente da un nome. Non essendo nel FileSystem presente un astrazione delle directory, il nome è l'intero path al File.
I File supportano una compressione seamless, opzionalmente attivabile dal file di config. Se questi vengono impostati come compressi, infatti, sulle write/append comprimeranno automaticamente i propri contenuti. Nelle read, invece, li decomprimeranno nel buffer del risultato. La compressione è implementata dalla libreria zlib.

Se un operazione farebbe superare al FileSystem i limiti della capacità stabiliti all inizializzazione, non la esegue, restituendo EOVERFLOW in errno. Quando accade ciò il chiamante dovrà occuparsi di chiamare freeSpace() specificando quanto spazio liberare, causando una capacity miss.
Le capacity miss sono gestite quindi all interno di freeSpace, che farà una chiamata alla policy per la scelta di un bersaglio da eliminare. Questo prenderà sempre il primo File (non locked) dalla List dei metadati, che verrà però prima ordinata secondo un euristica appropriata a seconda della policy scelta in configurazione. Tramite questo meccanismo sono supportate policy arbitrarie.

\subsubsection{ThreadPool}

La struttura ThreadPool deve tenere traccia di:
\begin{description}

\item[Le task da eseguire.] Vengono conservate in una SyncQueue, una coda sincrona che permette ai thread di venire deschedulati mentre attendono nuovi elementi.

\item[I thread vivi.] Tiene il numero dei thread vivi in un AtomicInt, e i loro pid in una List. Queste sono aggiornate automaticamente dai worker thread alla loro nascita e morte.

\end{description}

La terminazione dei worker thread è gestita prima di tutto con la flag terminate. Questa viene controllata da ogni worker thread prima e dopo l'aver estratto una task dalla SyncQueue. In quanto thread in attesa di nuovi lavori nella SyncQueue sono però deschedulati, vengono anche inseriti nella pool dei lavori fittizi die(), che causano la terminazione del thread chiamante, svegliando quindi e terminando tutti i thread in attesa.
Si supporta inoltre la cancellazione dei worker thread, per quanto questa sia sconsigliata in quanto rischia di portare alla perdita di risorse. Questa viene chiamata nella FastExit come ultima risorsa, se i thread non terminano in modo pulito entro 5 secondi.

\subsubsection{ConnState}

La struttura ConnState deve tenere traccia di:
\begin{description}

\item[I file aperti.] I FileDescriptor restituiti dal FileSystem vengono conservati in un HashTable con chiave il nome del File.

\item[Il file locked.] Il FileDescriptor di un eventuale File in stato locked viene anche conservato separatamente, in modo da poter implementare un limite ad un singolo File locked per connessione.

\end{description}
All apertura  di un File, il FileSystem genera un FileDescriptor. Questo associa al nome del File delle flag, che segnano quali permessi l'utilizzatore ha su di questo (lettura, scrittura, se il File è lockato o meno). Per successive operazioni sul File dovremo poi usare questo FileDescriptor.
\\
È stato scelto di limitare il numero di File locked in un determinato momento a uno per connessione. Quando quindi una connessione va a chiedere la lock di un File, implicitamente richiede la unlock di eventuali File che tenesse locked.
Questa scelta implementativa ci è sembrata ragionevole e giustificata per una serie di motivazioni.
Prima di tutto, nell API come dato nel testo del progetto, è implicito che due client possano mandarsi in deadlock a vicenda. Se infatti il ClientA locka prima file1 e poi file2, e ClientB locka prima file2 e poi file1, per le specifiche dell API si deve andare in deadlock, almeno che non si possano rilasciare File già lockati. Questo è quello che facciamo, rilasciando sempre il file già locked, limitando quindi i client a un solo file locked alla volta, e risolvendo qualsiasi situazione di deadlock tra client.
Inoltre, sono pochi i workflow in cui sarebbe ragionevole ottenere la lock su più file in contemporanea, mentre qualsiasi workflow rischierebbe di andare in deadlock senza questa limitazione.
Ci sembra una buona idea dare agli utenti una limitazione chiara se gli evitiamo così i rischi di malfunzionamenti ben più opachi. 

\section{Comunicazione Client/Server}

La comunicazione client/server accade tramite socket AF\_UNIX.
Sopra questi socket vengono serializzati e deserializzati una serie di strutture, prima di tutte il Message.
I Message rappresentano alternativamente una richiesta dal client al server, o una risposta dal server al client. Contengono due valori numerici: Un tipo, che contiene il tipo della richiesta (ad esempio di read o di write),e uno status, che contiene l'esito della risposta (OK o un qualche tipo di errore).
Contengono inoltre una stringa null terminata info, in cui il server metterà messaggi informativi sul esito della richiesta, e che il client userà a seconda della richiesta, spesso per specificare il nome del File bersaglio. Infine possiedono un contenuto, un array di bytes che può essere usato a seconda della richiesta, o in risposta per mandare contenuti di file in seguito a una read o a un capacity miss.

Per mandare più File in un solo Message, come a seguito di una capacity miss o di una readN, esistono i FileContainer. Questi associano semplicemente a un contenuto opaco un nome. Sono però presenti metodi per serializzare e deserializzare array di FileContainer in un singolo buffer, permettendo di mandare un numero arbitrario di File in un unico Message, senza perderne il nome.

\section{Migliorie Possibili}

Per quanto il sistema WinSome nella forma corrente sia perfettamente funzionale, e capace di gestire un traffico moderatamente alto, sono possibili notevoli migliorie nella sua UI e performance.
Un interfaccia grafica darebbe chiaramente un grande miglioramento della user experience.
Dal punto di vista della performance, poi, identifichiamo alcuni bottleneck notevoli che, in caso di traffico troppo elevato potrebbero diventare dei problemi.

Il primo e più notevole è il singolo ReadWriteLock su ServerData. In quanto ogni connessione condivide deve condividere l'accesso a questo, se qualunque di queste ottiene una lock di write andrà a bloccare ogni altra connessione nella durata della sua operazione. Ciò si potrebbe mitigare con una strategia di lock più granulare, avendo quindi lock separate per ogni struttura dati, al costo di inserire una certa complessità nel codice. La soluzione migliore sarebbe in realtà banalmente di usare un sistema di database esistente come struttura dati di supporto, e ciò ci darebbe certamente un efficienza maggiore rispetto a una soluzione creata da noi.

Inoltre, l'implementazione delle comunicazioni TCP tramite threadpool va a creare un thread separato per ogni connessione che viene gestita. Questo è inefficiente se confrontato ad esempio a una soluzione che usi NIO e il multiplexing di canale, che andrebbe invece a usare un numero fisso di thread per gestire le connessioni in arrivo, con costi di context-switching notevolmente minori.
Queste due migliorie permetterebbero probabilmente al server di gestire traffici molto più elevati in futuro.

Detto questo, non crediamo necessarie al momento eccessive ottimizzazioni, che ci paiono premature rispetto al numero di utenti attivi (0).

\section{Codice di terze parti}
Per l'implementazione dell HashTable utilizziamo la funzione hash Murmur3 sviluppata da Austin Appleby e hostata su github al link \url{https://github.com/aappleby/smhasher}. Questa è rilasciata nel dominio pubblico senza copyright.

Utilizziamo inoltre la libreria zlib per la compressione dei File. Questa è copyright di Jean-loup Gailly e Mark Adler, che ne permettono l'utilizzo libero. Il progetto è trovabile a \url{https://www.zlib.net/}.

Infine, per generare i nostri UUID, utilizziamo un frammento di codice preso da stackoverflow cortesia dell utente themoondotshine all indirizzo \url{https://stackoverflow.com/a/2182269}.

\end{flushleft}

\end{document}

